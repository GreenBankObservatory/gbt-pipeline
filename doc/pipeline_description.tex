\documentclass[11pt]{article}
\begin{document}
\title{KFPA IDL Prototype Pipeline Description}
\author{Joe Masters}
\date{April 21, 2010}
\maketitle

\begin{abstract}
The purpose of this document is to describe the IDL KFPA Pipeline code provided by Glen Langston for the purpose of calibration and imaging multi-beam, K-band datasets from the Green Bank Telescope.

In particular, this description applies to github commit \#a26a9777b9, which is installed in /home/sandboxes/kfpa\_pipeline (since 2010-04-14), but it also applies to the original pipeline code supplied by Glen.

All descriptions below apply to a single spectral band (IF), feed and polarization.  The pipeline, however will process all bands, feeds and polarizations in an map.

A map consists of a reference scan, followed by one or more mapping scans, followed by a second reference scan.
\end{abstract}

\section{Avegerage elevation and system temps for reference scans}

For each reference scan we:

\begin{enumerate}

\item Average the CALON and CALOFF integrations as:

\begin{equation}
CAL_{ave} = (CALON_{ave} + CALOFF_{ave}) / 2
\end{equation}

where $CALON_{ave} = (\sum\limits_{i=0}^n i_{CALON}) / n$ and $CALOFF_{ave} = (\sum\limits_{i=0}^n i_{CALOFF}) / n$.

\item Average the elevation for all CALON integrations.

\item Compute the total exposure time for all CALON integrations.

\item Compute $T_{sys}$ as:

\begin{equation}
T_{sys} = avg( CAL_{ave}[80] / (CAL_{diff})[80] ) * TCAL_{mean}
\end{equation}

where $[80]$ represents the center 80\% of the bandpass and $TCAL_{mean}$ is the average off all CALOFF TCAL values for the scan.

\end{enumerate}

\section{Average cals over all map scans}

For the full block of map scans we difference the CALON and CALOFF integrations as:

\begin{equation}
CAL_{diff} = CALON_{ave} - CALOFF_{ave}
\end{equation}

\section{Compute zenith opacity tau ($\tau$)}

Next we retrieve the archived $\tau$ prediction for date and frequency using Ron Maddalena's {\it getForecastValues} script, installed only in his home directory in Green Bank.

\section{Smooth and remove $T_{sky}$ from refs and map $CAL_{diff}$}

Smoothing of the averaged spectra and the averaged map block spectra are performed as follows:

\begin{enumerate}

\item Smoothing the averaged difference of CAL states ($CAL_{diff}$):

\begin{enumerate}
\item median smoothing applied to remove narrow band RFI, with a filter of size {\bf 5} channels.


\begin{equation}\label{rfi}
CAL_{ave\_median} = median(CAL_{ave},5)
\end{equation}

\item Savitzky-Golay smoothing is further applied.

\begin{equation}\label{sg1}
CAL_{ave\_savgol} = savgol(CAL_{ave\_median})
\end{equation}

\item Convert to units of K/Count:

\begin{equation}\label{kcount}
cal_{k/count} = TCAL_{mean} / (CAL_{diff})[80]
\end{equation}

\item Savitzky-Golay smoothing with different parameters is again applied to band edges.

\begin{equation}\label{sg2}
CAL^{*}_{ave\_savgol} = CAL_{ave\_savgol} * cal_{k/count}
\end{equation}

\item Boxcar smoothing is applied to the spectrum to reduce effects of joining previous two Sav-Gol filters.

\end{enumerate}

where $[80]$ represents the center 80\% of the bandpass and $TCAL_{mean}$ is the average off all CALOFF TCAL values for the scan.

\item Smoothing the average of CAL states ($CAL_{ave}$):
\begin{enumerate}

\item median smoothing applied to average of reference spectrum integrations to remove narrow band RFI, with a filter of size {\bf 5} channels as in equation (\ref{rfi}).

\item Savitzky-Golay smoothing is further applied as in equation (\ref{sg1}).

\end{enumerate}

\item Convert smoothed $CAL_{ave\_savgol}$ to K/count units as in equations (\ref{kcount}) and (\ref{sg2}).

\item Determine sky temperature contribution to spectrum ($T_{sky}$).
This is dependent upon the observed frequencies and date of observation.

\item Subtract $T_{sky}$ contibution from $CAL_{ave\_savgol}$ and median smooth (cell size {\bf 11}).

\begin{equation}
CAL^{*}_{ave\_savgol} = median(CAL_{ave\_savgol} - T_{sky},11)
\end{equation}


\end{enumerate}

\section{Apply gain correction to all map integrations}

\begin{enumerate}

\item Gather UTC times from reference scans to later be used for interpolation.

\item For each scan and each integration:

\begin{enumerate}

\item Get the frequency range of each channel.

\begin{equation}
freq_{start} = CALON.observed_frequency + ( (0-refChan) * delChan ) * 1.E-6
\end{equation}

\begin{equation}
freq_{end} = CALON.observed_frequency +  ( (nChan-refChan) * delChan ) * 1.E-6
\end{equation}

\item Get opacity tau ($\tau$)

\item Determine sky temperature contribution to integration ($T_{sky}$).

\item Correct for opacity per channel

\item Average CALON and CALOFF (CALONOFF)

\item Compute and interpolated reference from the two reference scans and also interpolate $T_{sys}$. ($ref_{interp.}$)

\item Update inegration spectra.

\begin{equation}
CALON(\nu) = opacity(\nu)* ((CALONOFF * CAL_{diff}) - (T_{sky}(\nu) + ref_{interp.}))
\end{equation}

\item Update $T_{sys}$

\begin{equation}
CALON(\nu).tsys = CALON[80] + T_{sky}[mid] + CAL_{diff}.tsys
\end{equation}

\end{enumerate}

\item Convert to AIPS friendly format (idlToSdfits) and image.

\end{enumerate}

\end{document}

