%Description of the GBTIDL calibration process along with summary
%of AIPS mapping functions.
%HISTORY
%10JAN23 GIL Expand calibration info
%09DEC23 GIL Expand AIPS info
%09DEC18 GIL Expand on the observing details
%09DEC17 GIL first complete version
%09DEC16 GIL limit document to calBand calibration full description
%09NOV29 GIL expand imaging procedure description
%09NOV27 GIL initial version of test obs
%09AUG11 GIL clean up equations and expand for full formalism
%09AUG10 GIL revised latex version
%07JAN30 J?B initial version

\documentclass[12pt,twoside]{article}
%\usepackage[ascii]{inputenc}
\usepackage[T1]{fontenc}
\usepackage[english]{babel}
\usepackage{amsmath,amssymb,amsfonts,textcomp}
\usepackage{color}
\usepackage{calc}
\usepackage{longtable}
\usepackage{hyperref}
 \usepackage{graphicx}  
\hypersetup{colorlinks=true, linkcolor=blue, filecolor=blue, pagecolor=blue, urlcolor=blue}
\newcommand\textsubscript[1]{\ensuremath{{}_{\text{#1}}}}
% Text styles
\newcommand{\myitem}[1]{\item{\makebox[1.25in][l]{{\tt #1}}}}
\newcommand\textstyleInternetlink[1]{\textcolor{blue}{#1}}
\newcommand\textstylePageNumber[1]{\textcolor{black}{#1}}
\newcommand\textstyleFootnoteSymbol[1]{\textsuperscript{#1}}
% Outline numbering
%\setcounter{secnumdepth}{0}    %% use this for no section numbering
\setcounter{secnumdepth}{2}
\newcommand{\arcs}{''~}
\newcommand{\arcm}{'~}
% List styles
\newcommand\liststyleWWviiiNumii{%
\renewcommand\theenumi{\arabic{enumi}}
\renewcommand\theenumii{\arabic{enumii}}
\renewcommand\theenumiii{\arabic{enumiii}}
\renewcommand\theenumiv{\arabic{enumiv}}
\renewcommand\labelenumi{\theenumi.}
\renewcommand\labelenumii{\theenumii.}
\renewcommand\labelenumiii{\theenumiii.}
\renewcommand\labelenumiv{\theenumiv.}
}
\newcommand\liststyleWWviiiNumi{%
\renewcommand\theenumi{\arabic{enumi}}
\renewcommand\theenumii{\arabic{enumii}}
\renewcommand\theenumiii{\arabic{enumiii}}
\renewcommand\labelitemi{[F0B7?]}
\renewcommand\labelenumi{\theenumi.}
\renewcommand\labelenumii{\theenumii.}
\renewcommand\labelenumiii{\theenumiii.}
}
% Pages styles (master pages)
\makeatletter
\newcommand\ps@Standard{%
\renewcommand\@oddhead{}%
\renewcommand\@evenhead{}%
\renewcommand\@oddfoot{KFPA Calibration/Mapping Test Report \hfill \today}%
\renewcommand\@evenfoot{\@oddfoot}%
\setlength\paperwidth{8.5in}\setlength\paperheight{11in}\setlength\voffset{-1in}\setlength\hoffset{-1in}\setlength\topmargin{1in}\setlength\headheight{12pt}\setlength\headsep{0cm}\setlength\footskip{12pt+0.461in}\setlength\textheight{11in-1in-0.5in-0cm-12pt-0.461in-12pt}\setlength\oddsidemargin{1.25in}\setlength\textwidth{8.5in-1.25in-1.25in}
\renewcommand\thepage{\arabic{page}}
\setlength{\skip\footins}{0.0398in}\renewcommand\footnoterule{\vspace*{-0.0071in}\noindent\textcolor{black}{\rule{0.25\columnwidth}{0.0071in}}\vspace*{0.0398in}}
}
\newcommand\ps@FirstPage{%
\renewcommand\@oddhead{}%
\renewcommand\@evenhead{}%
\renewcommand\@oddfoot{}%
\renewcommand\@evenfoot{}%
\setlength\paperwidth{8.5in}\setlength\paperheight{11in}\setlength\voffset{-1in}\setlength\hoffset{-1in}\setlength\topmargin{1in}\setlength\headheight{12pt}\setlength\headsep{0cm}\setlength\footskip{12pt+0cm}\setlength\textheight{11in-1in-1in-0cm-12pt-0cm-12pt}\setlength\oddsidemargin{1.25in}\setlength\textwidth{8.5in-1.25in-1.25in}
\renewcommand\thepage{\arabic{page}}
\setlength{\skip\footins}{0.0398in}\renewcommand\footnoterule{\vspace*{-0.0071in}\noindent\textcolor{black}{\rule{0.25\columnwidth}{0.0071in}}\vspace*{0.0398in}}
}

\makeatother
\pagestyle{Standard}
\thispagestyle{FirstPage}
\setlength\tabcolsep{1mm}
\renewcommand\arraystretch{1.3}
% footnotes configuration
\makeatletter
\renewcommand\thefootnote{\arabic{footnote}}
\makeatother

\title{KFPA Test Observations on 2009 November 27}

\begin{document}
\clearpage\pagestyle{Standard}
\thispagestyle{FirstPage}
{\centering\bfseries
KFPA Pipeline Calibration and Mapping Procedures for \\ Observations on 2009 November 27
\par}


\bigskip

\centerline{Glen Langston}
\centerline{2010 January 24}

\bigskip

\setcounter{tocdepth}{3}
\renewcommand\contentsname{}

\bigskip
\abstract{This document describes the observations made with the GBT towards
galactic radio source W49.   We summarize the observations and present a
series of data calibration imaging steps to produce a set of four images
of this region, attempting to detect a variety of molecular species.
These observations were made using the dual feed, dual polarization GBT
high K-band (22 to 26 GHz) system, for spectral
line observations of molecular transistions of $NH_3$ $1-1$, $2-2$, $3-3$,
$H_2O$ and $CH_3OH$.   These observations were  
made in preparation for KFPA commissioning.   Poor weather prevented
us from achieving all the test goals, but a number of valuable lessons from this
session are discussed.   We summarize GBTIDL scripts that allow nearly completely
automatic calibration of dual beam observations.   We also outline requirements
for improved "Sub-mapping" procedures, to facilitate efficiency when
mapping large angular regions or observations interrupted for adverse weather.

This document contains a description of the major steps of
the calibration and mapping process and documents the RMS noise and
and Signal-Noise-Ratio achieved for the method of "Noise Diode Calibration".
In addition to test observations of W49, these calibration routines have been
tested on wide bandwidth
line searches for neutral hydrogen clouds near galaxy NGC 784.   

Later documents will compare different methods of calibration, but the implementation
of these alternate methods will not significantly change our approach to pipeline
calibration and imaging.
}

\tableofcontents
\clearpage
\begin{figure}
\vskip -1.5in
\includegraphics[width=150mm]{w49_nh3_3-3.pdf}
\vskip -1.25in
\caption{Image produced from the two beam, $NH_3$ $3-3$, observation of W49.
The region of peak emission was imaged with both beams.}
\end{figure}

\section{Test Goals}
The purpose of these tests were to prepare for commissioning of the the 7 pixel GBT
K-band focal plane array.   The KFPA will be normally be used in a rapid scan and 
dump of spectra mode, so we focused on mapping observations 
with the maximum spectrometer
dump rate.
These particular observations were intended to compare mapping and calibration 
methods for three observing methods, $1)$ position switched,
$2)$ position switched without winking cal and
$3)$ frequency switched.
We intended to make small maps of a point source, 3C286, and a larger map
of a known spectral line source, W49.   The point source maps were to be
8\arcm square and the spectral line maps were to be 15\arcm square.
Astrid scripts were prepared for these observing methods, but only
the position switched scripts were used. 
Poor weather allowed us to complete only part of the position switched observations.
We attempted a beam map of point source 3C286, 
but we were stopped for high winds.   After the winds,
snow accumulation required pointing the telescope
at lower elevations and we immediately moved to W49.

Only the position switched data were obtained during these tests, and future observations
must be scheduled to perform the frequency switched tests.
Data were obtained to allow position switched and noise-diode calibration.
In this document, we describe only the noise diode calibration method.
The conversion and imaging steps described here are applicable to all
calibration methods.
We intend to separately perform position switched calibration and compare
the calibration accuracy and spectral baseline flatness in a separate memo.

Section \S 2 describes the observations and \S 3 describes the calibration process.
Section \S 4 describes the conversion from GBTIDL {\tt keep} format to AIPS input
format. 
Section \S 5 describes the commands used in AIPS to examine and image the
observations.   Section \S 6 lists recommendations to ASTRID that will facilitate
mapping large regions and also maps interrupted for adverse weather.
Section 7 summarizes the test results.

Appendix A describes the highest level GBTIDL calibration procedure.   Appendix
B describes the noise diode calibration procedure.  Appendix C is an observing
log, to facilitate others to compare reduction processes.
Appendix D lists an AIPS {\tt runfile} to aid in configuring the AIPS imaging routines.

\section{Observations}

This document summarizes Mapping Test observations made with the GBT
on November 27, 2009 and the current 22 to 26 GHz receiver system.  
The observations were made immediately after the Thanksgiving
holiday shutdown, an NRAO holiday.   Kevin Gum first moved the
telescope slowly in Azimuth to condition the grease and bearing oil before
the observations.
 
Beams 3 and 4 were configured for four spectral bands, each with dual 
polarization and 50 MHz bandwidth.   This is the same spectrometer configuration
that will normally be used for the KFPA, 7 beam configuration.  
These observations tested the capability of the spectrometer
to dump multiple spectra at high rates.   

The spectrometer was first configured for 0.5 second integrations, but
we were unable to take data with the spectrometer in this mode.\footnote{During a maintenance period the spectrometer configurations will
be more fully tested with shorter integrations and non-winking cal signals.}
We changed to 1 second integrations with winking cal, toggling at a rate of 20 Hz.

The four spectral bands were configured for four groups of lines.
The $NH_3$ 1-1 and $2-2$ were in the first spectral band.
The second spectral band was configured for $H_2O$.   
The third spectral band was configured for $CH_3OH$ and $CCCS$,  and
the fourth spectral band was configured for $NH_3$ $3-3$.

The data were archived under project T\_09NOV27, and consisted of 130 scans
taken in the interval between 14:00 and 18:08 UTC.   Sources 3C286, J1856+06,
and the region surrounding W49 were observed.
 
\subsection[K-Band Mapping Sequence]{K-Band Mapping Sequence}
\bigskip

The K-Band mapping parameters are set by the beam size and the
sampling rate of the GBT spectrometer.   The fastest demonstrated
spectrometer dump rate is 1.0 for Cal On/Off observations for 16 spectral
bands.    The GBT FWHM beam at 22 GHz is 33\arcs, and for optimum sampling
the sky every 1/3 of a beam or approximately every 10\arcs in the along
telescope motion direction.
For these observing parameters, the telescope moves at a rate of 10'/minute.
Because the sky will be repeatedly observed with all beams, it is sufficient
to space the scans of observations by 1/2 beam.    
For a square image with 15\arcm sides, the observing time is 90 seconds
per scan and 60 scans are required.   For a 15\arcm square map, the total observing time is 90 minutes, plus point, focus and calibration time.   This duration is too long to expect
reliable focus and pointing offsets, without periodic peak and focus observations.

For the purposes of these tests, the 15\arcm map was divided into three
equal sections, offset in declination.   Before and after each observation a
reference location was observed in spectral line mode.

The requirement of halting mapping for point and focus checks, combined
with weather stoppage resulted in fairly poor observing efficiency.   


\section{Calibration Steps}

The observations were reduced with GBTIDL procedures.   These procedures were
built on exiting procedures.   The procedures are described in the appendix and 
summarized here.

\begin{description}
\myitem{w49}  The main procedure in the data reduction script is a fairly short
script that complies the required procedures and specifies the scans to be used for
reference calculations.   
\myitem{calBand} Main computational procedure processes all observations of one
spectral band.   The procedure calls sub procedures to compute calibration spectra
for both polarizations.
\myitem{getRef} Compute the reference spectrum and the average, smoothed
Cal-On - Cal-Off spectra for each beam, polarization and frequency band (molecular line).
\myitem{scaleRef} Perform the initial scaling and smoothing operations on the reference
and calibration spectra.   The outputs of this procedure are a pair of "data containers"
fully describing the spectra.
\myitem{calScanInt}  Procedure to fully calibrate a scan, separately processing
each integration and interpolating the reference spectra.   This procedure
uses the outputs of {\tt scaleRef}. The model sky contribution
to the system temperature is subtracted from the spectra.
\end{description}

The calibration routines generate an output file containing one or more polarizations,
but from only 1 GBT beam.   For this example, processing of the 23870 MHz observations
from the third of four beams of the GBT K-band feed, the calibrated W49 data
have name:  {\tt TCal\_NH3\_W49\_3\_51\_82\_23870.fits}
These files are given names with the following parts:
\begin{description}
\myitem{Calibration Type} Method of calibrating the Spectrum.  In this case the
calibration is performed only using the difference of the Noise Diode Cal-On - Cal-Off
values.
\myitem{Molecule} The name of one molecule in the spectrum.
\myitem{Source} Source name for the first scan in the observation.
\myitem{Beam} The beam of the observation.   In this case beams 3 and 4 (range 1 to N).
\myitem{First Scan}  First scan number processed.
\myitem{Last Scan}  Last scan number processed.
\myitem{Center Freq.}  Center frequency of this frequency band (MHz).
\end{description}

Before proceeding further, we must point out that 
all calibration of Radio Astronomy data depends on some method for removing
the receiver contribution to the system temperature and the spectral baseline shape.
In many cases, an off source position can be found, where it is reasonable to assume
there is no source contributing to the measured spectra.  This {\tt reference} position
may be subtracted from the observation to yield a good estimate of the spectral
properties of the science target spectrum.   However for some spectral lines,
such as $H_I$, there is no emission free off location.   Also since the GBT is
remarkably sensitive, there is also almost no location free of a background source,
so there is always some contribution to the spectral baseline from the integrated
collection of faint background sources.

Combined with an interest in attempting calibration from first principles, here
we have focused on noise diode only calibration.   The system 
contribution is still present in this reduction technique, and in the
next step the system contribution is removed by assuming that most
of the spectral baseline is free of line emission, and the baseline
is removed using a median filter baseline (See Langston and Turner 2007, {\it Ap. J.},
{\bf 658}, 455 for more details on the median filter baseline removal process).

\section{Conversion to {\tt AIPS} Input Format}

After completing the calibration process, the data are converted into another data
format compatible with the AIPS package.   The program {\tt idlToSdfits}, written by
Glen Langston, is used to convert the GBTIDL "keep" format files into the AIPS format.
The program has a variety of control parameters, which allow data flagging, data
selection, averaging of channels and subtraction of a median baseline.

The program {\tt idlToSdfits} takes as input the calibrated spectra and produces
an {\tt AIPS} "Single Dish Fits" format file, which is different that the GBT format.
The program gives this file the extension "{\tt .sdf}".   
By default, the program gives the output file a name similar to the input file.
The program also
produces a {\tt .tex} summary of the observation and a tabulation of the
RMS noise in all spectra  ({\tt .noi}).
The high level GBTIDL procedure automatically calls {\tt idlToSdfits} at the end
of calibration.

The selection of the spectra are selected and the spectral baseline removed by a
call from GBTIDL to a procedure {\tt toaips}.  This procedure accesses the 
output GBTIDL file name ({\tt !g.line\_fileout\_name}) and converts this
output into the required {\tt AIPS} input format.

\section{AIPS Processing}

The AIPS program is used to produce images for the observer.   This process
involves several steps.
Each AIPS task has a number of input arguments.   
Generally, the AIPS default values are good for most tasks, so if uncertain
about any value, try to set it to the default value.  The default values are
generally zeros or blank strings ('').

Each task has one or more input and output files.   AIPS also has the
concept of input {\tt DISKS} which can be thought of as different
output directories.   More information on AIPS can be found at
{\it http://www.nrao.edu/aips/}

\subsection{Loading data into AIPS: {\tt UVLOD}}
The data are first loaded into AIPS using the AIPS task {\tt UVLOD}.   
The most important input items are listed below.
\begin{description}
\myitem{ DATAIN}  Ascii string specifying the data location and file name.
Often the AIPS data are placed in the directory were AIPS is started.  
The current directory has a symbolic name {\tt PWD}.   
In the case of the W49 example, the input file name is 
{\tt PWD:W49\_3\_5\_82\_23870.sdf}.
\myitem{ DOUVCOMP}  The AIPS data is compressed by default, but
single dish data must not be compressed, so always use {\tt DOUVCOMP = -1}.
\end{description}

After loading the observations from several beams, the data must be merged
using the task {\tt DBCON}.
This task only allows two inputs, so the task must be run repeatedly to
merge all beams.   Remember to give the input data equal weight by setting
the AIPS parameter ({\tt REWEIGHT=0}.

\subsection{Mapping Parameters: {\tt SDIMG}}

The AIPS mapping task, {\tt SDIMG}, has a larger number of input parameters.  The
critical ones are listed below:
\begin{description}
\myitem{BCHAN} Beginning channel,  usually {\tt BCHAN=0}, is a good default,
but can potentially generate a large amount of unused image planes.
\myitem{ECHAN} Same as for {\tt BCHAN}
\myitem{CELLSIZE}  Size of individual image pixels.   The {\tt CELLSIZE} is a two
dimensional value, but usually the pixels are square.   A good rule of thumb is to set
the {\tt CELLSIZE} to 1/3 to 1/5 the GBT beam size at the center frequency.
For comparing observations of lines at several frequencies it is convenient to 
make all images with the same cell size and image size.    AIPS has the
capability of creating procedures for repeating tedious operations.  
Several AIPS procedures are listed in Appendix D for setting the
proper values for the imaging convolution step.

For the $NH_3$ $1-1$ transition at 23,694.506 MHz, the GBT FWHM beam size is
32.1".   Setting {\tt CELLSIZE=5} yields a reasonable image.
\myitem{IMSIZE} Image size in pixels.  The product of image size and {\tt CELLSIZE} in 
(arc-seconds per pixel) and should be slightly larger than the region mapped.
\myitem{XTYPE} The convolving function for placing spectra into the image grid.
The currently favored convolution type is {\tt XTYPE = -16}, indicating a circular
convolution shape that is the product of the first order Bessel function multiplied by an exponential.   The convolution function is given in the equation below for angular 
offset $\theta$:
$$
f(\theta) = e^{- [\theta/Xparm(3)]^{Xparm(4)}} ~ Bessel_{J=1}( \theta/Xparm(2)) / (\theta/Xparm(2))
$$ 
\myitem{XPARM}  The convolving function $f(\theta)$ has three parameters that are specified
in units of arc-seconds, and the forth parameter, the exponential exponent, is
always 2.   These convolving function parameters are set relative to the single dish
beam size and may conveniently set using an equation in AIPS.   
An AIPS procedure {\tt SETCONV} is listed in appendix D, to aid in the setup.
This procedure is summarized here.  First an AIPS variable, {\tt BMAJ}, is defined
by an equation to be approximately the GBT beam size (in arc seconds) at the 
rest frequency of the line.
All other parameters in the convolution function are multiples of this value:
\\
{\tt XPARM =  BMAJ, BMAJ*1.55, BMAJ*2.55, 2} ~~~

The first {\tt XPARM} argument is the angular size of the convolving function.
The inquisitive observer can get more {\tt AIPS} help on this topic by typing {\tt HELP UV6TYPE}.
\myitem{REWEIGHT} The data of low significance may be flagged using the
{\tt REWEIGHT} parameter, but usually this produces unsatisfactory results.  To
avoid this, use {\tt REWEIGHT = 0, 1E-12}.   (Remember to set this value
back to zero before running {\tt UVLOD} again!, or the second data set will have
insignificant weight).
\myitem{OPTYPE} The project on coordinates onto a rectangular grid is controlled by
this parameter.  For large single dish fields,  setting {\tt OPTYPE = '-GLS'} produces
a rectangular image.
\end{description}
\subsection{Examining the Image: {\tt TVLOD, TVPS, TVLAB, TVWED, TVWLAB}}
To load images for examination, list the AIPS image catalog ({\tt MCAT}) and
select the image ({\tt GETN ??}), where {\tt ??} is the number of the
image.     First set the pixel range to all values, {\tt PIXRA = 0} and load
the image {\tt TVLOD}.   To label the image, type {\tt TVLAB} and
show a color wedge type {\tt TVWED}.   To label the color wedge
type {\tt TVWLAB} and then to adjust the colors type {\tt TVPS}
and use the mouse and keyboard letters {\tt A, B, C and D} to 
adjust the color range.

An AIPS procedure {\tt TVLOOP} is listed in Appendix D, which is
convenient for rapidly viewing selected channels of the resulting image.

\subsection{Spectrum of an Image Region: {\tt ISPEC}}
After producing the map, the spectral properties of a region may be
examined using AIPS task {\tt ISPEC}.   Generally ISPEC is used several
times to select the region of interest.   First select the image to examine,
load it on the AIPS TV, using {TVLOD} and adjust the colors using
{\tt TVPS}.   After setting the color range to show the noise
and regions with emission,  use {\tt TVWIN}  to select a large region, 
avoiding the partially mapped regions.    Then set the display to 
the AIPS TV ({tt DOTV=1}) and do not print out the values ({\tt DOCRT=0}).
Set the axis value to channels (pixels) {\tt LTYPE=6}, so that
the when running {\tt ISPEC}, the channels containing emission can
be selected for production of a moment map.
\subsection{Switching Frequency/Velocity 3rd Dimension}
The GBTIDL calibration has transfered the rest frequency of this observation
to the file header.  With that information, AIPS can transform between velocity
and frequency on the 3rd axis.  The AIPS verb {\tt ALTSW} switches
between these two options.   Running {\tt ALTSW} a second time
switches back to the original display option.
\subsection{Image of Molecular distribution: {\tt TRANS} and {\tt MOMNT}}
Often the observer will need to view the distribution of a particular
molecular transition over all velocity ranges.   Two AIPS tasks are
required to accomplish this.   The first is {\tt TRANS} which is
used to re-organize the image from RA, Dec, Frequency order to
Frequency, RA, Dec order.    This new order is specified
by the argument {\tt TRANSCOD = '312'}. 
At the same time, the image planes containing emission may be selected using the 
{\tt Bottom Left Corner (BLC)} and {\tt Top Right Corner (TRC)} arguments.
The task will run much faster if only the required channels are selected, so
set the {\tt BLC(3)} and {\tt TRC(3)} values to the start and stop channels
with line emission.   

After transforming the image, an new {\tt TRANS} image is produced.  Select
this image for input to {\tt MOMNT}, using {\tt GETN ??}.   If only the total
intensity image is required, set the output class to {\tt OUTCL = '0'}, and
to include all values in the selected region {\tt ICUT= 0}.
Produce the total intensity image by commanding {\tt GO MOMNT}.
Find the new image ({\tt MCAT}), select it ({\tt GETN ??}) and display it
({\tt PIXRA 0; TBLC 0; TTRC 0; TVLOD})

\subsection{Measure Image properties: {\tt JMFIT}}
The source structure and integrated intensity is measured using {\tt JMFIT}, but
first the telescope resolution must be described in the image header.   Above,
the AIPS procedure {\tt GBTBEAM} is listed.   
{\tt GBTBEAM} will set this information in the AIPS image header.    
The value is approximate, as the mapping method can reduce the effective
angular resolution of the observations.   The most precise method of determining
the angular resolution is mapping a point calibration source in the manner
identical that used for making the science observation.

After setting the beam size, the angular region to be fit is selected with
AIPS verb {\tt TVWIN}.    The number of iterations of the fitting should
be large, {\tt NITER=1000}.   The number of gaussian components to
fit depends on the source structure.   For a single component set
{tt NGAUS = 1}.   To fit position, width and intensity set 
{\tt DOPOS = 1; DOMAX = 1; DOWID = 1}.
Don't display the values on the terminal ({\tt DOCRT=0}).
  
\section{Observation Support Recommendations}
One important result of these tests is the identification of requirements for
improving the ASTRID
mapping bookkeeping to increase observer efficiency for restarting 
suspended map observations.  

\subsection{ASTRID Script Improvements for Mapping}
Below are items that would be beneficial for future mapping procedures.
The goal of these changes is to more optimally complete sections of maps with 
appropriate calibration, and booking log that facilitates restarting the maps.

\begin{itemize}
\item Observing procedures should track interval between point and focus observations.
\item Mapping procedures should automatically suspend mapping, then perform point, focus
and calibration.
\item Mapping procedures should resume mapping scans automatically after suspending
mapping for calibration or weather.
\item Mapping procedures should track remaining observing time and perform
the final calibration observations just before the end of observing sessions.
\end{itemize}

\subsection{Save/Restore Configurations}

The current configuration tool does a good job of completely setting up
the GBT RF/LO/IF/Data acquisition systems. 
A second valuable feature that should be added to the configuration tool is restoring
a previous GBT configuration, from a previous mapping session.  Having consistent
gain (attenuator) settings would be valuable for checking gain stability
and be potentially valuable for a more consistent calibration of a set of 
sub-images of a larger image.

\subsection{Expanded Spectrometer Bands}
Many molecular clouds exhibit numerous strong spectral lines, but the
current GBT spectrometer can only be configured to fully observe one spectral 
band with all beams.   Frequently the science goal for mapping, will
be aimed at weaker lines.  It will be valuable to configure the 8th pair of IF
bands for a single beam and set up for observations of a stronger molecular
line.   In addition, the GBT should also be configured to allow observations
of a third molecular line, by use of the GUPPI design.   The GUPPI design
would be valuable for galactic chemistry observations if it can be configured
for sufficient spectral resolution and the output spectral line data
can be quickly averaged and reduced in the existing spectral line systems.

After completing the initial implementation of the GUPPI spectrometer, additional
capability can be implemented incrementally, as more funds for
hardware becomes available.

\section{Summary}

The position switched observations of W49 were carried out successfully,
with the observations of a 15\arcm map divided into three sections.
All data were processed in a consistent manner with the calibration scripts.
The full calibration processing time is significant for a 2 beam, 4 spectral
band observation.  The processing required almost 24 hours for these
mapping observations.   The pipeline processing can be significantly
speeded if the calibration can be run on several processors simultaneously.

The time required for calibration, pointing and focus were underestimated
and future tests should image smaller regions.

These data reduction scripts produce reasonable images of the
observations.  We show that the Cal-On - Cal-Off observing method
can produce good spectral baselines, but only with a dynamic range of
about 1 in 400, relative to the system temperature.  This implies
that spectral features fainter than 20K/400 $\sim$ 0.05K can
not be reliable detected with this technique.
Note that if the observer can accurately identify line - free regions, then
AIPS can be used to fit a spectral baseline, after the imaging step.
This allows fainter lines to be detected.   Also note that frequently
a final spectral baseline must be subtracted in the position switched
and frequency switched calibration modes.

In a companion document, we describe the RMS noise found for
position switched and frequency switched observations.

Future KFPA images can be produced using these methods, but a
number of improvements to the starting and stopping of mapping
image sub-sections would improve observer efficiency.


\begin{figure}
\vskip -0.75in
\includegraphics[width=120mm,angle=-90]{n784SmoothCal.pdf}
\vskip -.75in
\includegraphics[width=120mm,angle=-90]{n784SmoothCalZoom.pdf}
\vskip -0.5in
\caption{Band pass spectrum of the smoothed Cal-On - Cal-Off signals and
model fit to the spectrum.  Bottom, zoom in on particularly sharp cusp
in the spectrum and the model fit.
}
\end{figure}

\clearpage
\appendix
\section{Calibration Procedures for Mapping}
This appendix describes in more detail the functions of the mapping
procedures and points at the relevant new features of these procedures.
Each mapping technique makes some assumptions concerning the
data quality and system stability.   The procedure methods which 
make these assumptions are high lighted.

\subsection{Top Level Procedure Example: w49.pro}
The GBTIDL mapping calibration procedures are all called by a single high
level GBTIDL script.  This script must be configured by the observer to
select the scans for calibration and for computation of the reference
spectra.   The reference spectrum has two distinct functions.  The first
and foremost is for use in gain calibration of the observation.  The second use
is for subtracting  of the electronics and sky contributions to the measure of
the sky intensity.   The gain calibration is always done by use of measurements of
the injected noise diode signals and laboratory measurements (or astronomical observations)
of the effective noise temperature of these noise diode values.


\begin{verbatim}
;IDL Procedure to calibrate map scans
;HISTORY
; 10JAN23 GIL use toaips to prepare data for AIPS imaging
; 10JAN22 GIL add code for selecting line emitting region
; 09DEC16 GIL break up sdfits call for clarity
; 09DEC15 GIL revised for tmc map
; 09DEC02 GIL revised for a 2x2 degree map
; 09NOV30 GIL initial version

@compilePipeline.pro

;The data can be loaded from inside idl, so that when the data are
;transformed to an sdfits file, they will be immediately calibrated
sdfitsStr = '/opt/local/bin/sdfits -fixbadlags -backends=acs'
;specify scan list, if all spectra are needed
scansList = ' '
;else specify only the desired scans
scansList = '-scans=49:83'
dataDir = '/home/archive/science-data/tape-0028/'
dataDir = '/home/gbtdata/'
projectName = 'T_09NOV27'
;From the Unix prompt type
sdfitsCmd = sdfitsStr + ' ' + scansList + ' ' + dataDir + projectName
;Tell observer what's being done
print, sdfitsCmd
; or spawn within IDL (uncomment the line below)
spawn, sdfitsCmd

mapDataName=projectName + '.raw.acs.fits'
filein,mapDataName

;now specify first and last scans, to guide mapping
firstScan=51
lastScan=82
refscans = [50,83]
;use all map scans as ref scans
allscans = indgen(1+lastScan-firstScan) + firstScan
refscans = allscans
print,allscans

; observation is for two feeds and two polarizations
nPol=2

; set velocity parameters for selecting relevant channels
vSource = 10.0        ; km/sec - defines center channel to select
vSourceWidth  = 10.0 ; km/sec - defines median filter width
vSourceBegin  = -30.0 ; km/sec - defines beginning channel to select
vSourceEnd    = 50.0 ; km/sec - defines endding channel to select
; The rest frequency frequencies guide the selection of data to be
; converted to AIPS format.   If no rest frequencies are provided,
; The rest frequencies in the observation header are used.
;            NH3 1-1 and 2-2, H2O,         CH3OH+CCCS,   NH3 3-3
restFreqHzs = [ 23694.5060D6, 22235.120D6, 23121.024D6,  23870.1296D6]
;below the line rest frequency for each band is set.
;There are many-many NH3 lines, so to set the velocity the strong line 
;must be identified.  

for iFeed = 0, 1 do begin $\
for iBand = 0, 3 do begin $\
  gettp,refScans[0], int=0, ifnum=iBand & $\
  calBand, allscans, refscans, iBand, iFeed, nPol & $\
  data_copy, !g.s[0], myDc & $\
;change rest frequency for computation of velocities in AIPS
  myDc.line_rest_frequency = restFreqHzs[iBand] & $\
;select channels and write the AIPS compatible data 
  toaips,myDc,vSource,vSourceWidth,vSourceBegin,vSourceEnd & endfor&endfor

\end{verbatim}
\centerline{Listing of the top level calibration script: w49.pro}

\clearpage
\begin{figure}
\vskip -0.75in
\includegraphics[width=80mm]{n784_smoothCal.pdf}
\includegraphics[width=80mm]{n784_smoothCalZoom.pdf}
\vskip -0.5in
\caption{Spectrum of galaxy NGC 784 observed with the GBT by Katie Chynoweth
and calibrated using the pipeline proceedurs.   The spectrum at left shows the
full range of intensities.  The strongest line is emission from the Milky Way, and
the second strongest line if NGC 784.   The spectrum at right
is identical to that on the left, except zoomed in on the spectral baseline.
The galaxy NGC 784 is clearly visible, but there is significant spectra
baseline wripple.  
}
\end{figure}

\section{Calibration Steps}

The observations were calibrated using previously written GBTIDL scripts.
The main script, {\tt w49.pro} has a number of steps.   These procedures
are currently written by the observer.  In the future, the pipeline will automatically
generate these scripts and run them on a set of processors.

\begin{description}
\myitem{\tt sdfits}  The main procedure will select relevant data from the
GBT archive and place these data in a file for calibration.   The linux program
{\tt sdfits} is called from inside IDL, so that calibration can start immediately
after the data selection.
\myitem{Scan Selection}  The calibration process requires identification of
scans appropriate for gain (reference) calibration.   The type of calibration
determines the scans to be selected.   For Noise Diode calibration, frequently
all scans in a observation will be used.   An exception is the case where the
source to be observed is so strong that the noise diode measurements are
adversely effected by the source.   This is the case for the W49 water maser
observations.    In this example, two off source reference spectra are combined
with the first and last map scans, to compute the reference Noise Diode spectra
for each beam and polarization.
\myitem{Band Selection}  This particular implementation of the calibration process
separately calibrates each spectral line observed.   This may be easily parallized
so that different processors calibrate different bands.    In the KFPA case, usually
there will be only one band, but multiple beams.   Potentially this script
should be re-written to separately identify different beams for calibration.
\myitem{Calibration}  The calibration process is completely performed by one
of three calibration processes.
\begin{description}
\myitem{\tt calBand}  Calibration of the gain using only the noise diodes, but performs
to subtraction of the receiver temperature contribution to the intensity values.
The model atmospheric contribution to the spectra is removed.
\myitem{\tt calBandRef} (Signal - Reference)/Reference calibration of the observations.
In this case a set of reference spectra are chosen which are free of line emission.
After the amplitude calibration, the reference spectra is subtracted.  
The model atmospheric contribution to the spectra is removed, by tracking changes
in the weather and antenna elevation between observations and the values at the
time of reference spectrum observation.
\myitem{\tt calBandFS} Frequency Switched calibration of the observations.   This technique
requires no specification of the reference signal, to be subtracted from the observation, but
does require specification of gain calibration observations.
\end{description}
\end{description}

\subsection{High Level Noise Diode Calibration: \tt calBand}

The Noise Diode calibration method is implemented in GBTIDL script
{\tt calBand.pro} which computes a noise diode spectrum for each beam and polarization
of the observations.   Only one spectral band is processed per call.   

\clearpage
\section{W49 Observing Log}

Below is a list of the observations during the W49 test.  Due to poor weather, we initially had trouble with 
the peak and focus observations, required a second set of peak and focus observations.
Note that probably the "Relaxed Fit" option should be the default for Astrid for
all KFPA observations.

The observations were suspended for high winds, then snow during the 3C286 Map 
interval.  After the winds decreased, we elected to observe W49, 
which was at low elevation, so as to minimize snow accumulation.
  
\begin{verbatim}
StartStop Source Proc.  # RA-2000  Dec-2000   #   Sky  BandWidth
   Scan   Name             (hms)    (d'")    Data (MHz)(MHz)
==== /home/gbtdata/T_09NOV27/ ==== T_09NOV27 ==== Glen Langston ========
   1   4 3c286   Peak   4 13h31m08 30d30'31  576 22105  990 DCR 14:02:24
   5   5 3c286   Focus  1 13h31m08 30d30'33  294 22105  990 DCR 14:04:52
   6   9 3c286   Peak   4 13h31m08 30d30'31  576 22105  990 DCR 14:08:25
  10  10 3c286   Focus  1 13h31m08 30d30'33  294 22105  990 DCR 14:10:53
  11  14 3c286   Peak   4 13h31m08 30d30'32  576 22105  990 DCR 14:14:49
  15  18 3c286   Peak   4 13h31m08 30d30'32  576 22105  990 DCR 14:20:02
  19  19 3c286   Focus  1 13h31m08 30d30'33  294 22105  990 DCR 14:22:31
  20  21 3c286   Nod    2 13h31m08 30d30'33  122 22237   50 ACS 14:32:52
  22  23 3c286   Nod    2 13h31m08 30d30'33  122 22237   50 ACS 14:37:48
  24  25 3c286   Nod    2 13h31m08 30d30'33  124 22237   50 ACS 14:45:14
  26  27 3c286   Nod    2 13h31m08 30d30'33  124 22237   50 ACS 14:51:26
  28  28 3c286Off Track 1 13h35m00 30d30'33   62 22237   50 ACS 14:53:02
  29  41 3c286   RAMap  ! 13h31m01 30d28'33  770 23700 1280 DCR 14:56:08
  42  45 J1856+06 Peak  4 18h56m32  6d10'15  720 23700 1280 DCR 15:31:50
  46  46 J1856+06 Focus 1 18h56m32  6d10'17  294 23700 1280 DCR 15:34:08
  47  48 W49     Nod    2 19h10m13  9d06'13  122 22234   50 ACS 15:38:48
  49  50 W49     Nod    2 19h10m13  9d06'13  122 22234   50 ACS 15:42:15
  51  51 W49Off  Track  1 19h14m13  9d06'13   61 22234   50 ACS 15:43:44
  52  82 W49    RAMap  31 19h10m13  9d08'42 2821 22234   50 ACS 15:47:11
  83  83 W49Off  Track  1 19h14m13  9d06'13   61 22234   50 ACS 16:46:22
  84  87 J1856+06 Peak  4 18h56m32  6d10'15  576 23700 1280 DCR 16:49:00
  88  88 J1856+06 Focus 1 18h56m32  6d10'17  294 23700 1280 DCR 16:51:18
  89  90 W49     Nod    2 19h10m13  9d06'13  122 22234   50 ACS 16:55:28
  91  92 W49     Nod    2 19h10m13  9d06'13  122 22234   50 ACS 16:58:55
  93  93 W49Off  Track  1 19h14m13  9d06'13   61 22234   50 ACS 17:00:25
  94 124 W49+7   RAMap 31 19h10m13  9d15'42 2821 22234   50 ACS 17:03:54
 125 125 W49Off  Track  1 19h14m13  9d06'13   61 22234   50 ACS 18:03:10
 126 129 J1856+06 Peak  4 18h56m32  6d10'15  576 23700 1280 DCR 18:05:53
 130 130 J1856+06 Focus 1 18h56m32  6d10'17  294 23700 1280 DCR 18:08:14
\end{verbatim}

\section{AIPS Utility for Imaging parameters}
Imaging of GBT spectra in AIPS is relatively straightforward, however
the setting the convolving function values to yield good results is
slightly tricky.   Below is a listing of an AIPS {\tt runfile} to
aid in setup of the inputs to AIPS task {\tt SDIMG}.

The critical procedure {\tt SETCONV}, below is run after all other inputs
to the AIPS task {\tt SDIMG} are set.   This includes selecting the input
spectra for the subsequent imaging process.   The header of the input
single dish data will be modified to include an estimate of the GBT beam size
at the frequency of the molecular line.

To use these procedures, within AIPS one must first compile these procedures.
to do this type 

{\tt RUN IDLTOSD}

at the AIPS command prompt.


Below is a listing of the {\tt IDLTOSD.001} procedure for this date.

\begin{verbatim}
$ aips scripts transfer circular beam shape to map header 
$ Allows imstat to return the correct intensities
$ HISTORY
$ 10JAN24 GIL add scripts for seting SDIMG convolution function

$ define some variables for procedures
PROC INITGBT
  SCALAR FREQMHZ, FWHM, BEAMSIZE, LAMBDACM
  SCALAR ISTART, IEND, PAUSEL
FINISH

$ Assume a "Circular Clean Beam"
$ INPUT beam size in arc-seconds
PROC CIRCULAR( X)
  keyword='BMAJ'   ; KEYVAL = X/3600., 0; PUTHEAD 
  keyword='BMIN'   ; KEYVAL = X/3600., 0; PUTHEAD 
  keyword='BPA'    ; KEYVAL = 0; PUTHEAD 
  keyword='NITER'  ; KEYVAL = 1, 0; PUTHEAD
  RETURN
FINISH

$compute the GBT beam size expected for a GBT mapping
$observation.   This model assumes the data are sampled
$much faster than the time to cross a point source, so
$no source smearing correction for finite integration
$time is required.
PROC GBTBEAM
  keyword 'restfreq' ; gethead
  FREQMHZ = KEYVAL(1)*.000001
  LAMBDACM = 29979.245/FREQMHZ
  BEAMSIZE = 0.423*LAMBDACM*60.
  CIRCULAR( BEAMSIZE)
  KEYWORD 'BUNIT'; gethead
  KEYWORD 'BUNIT'; KEYSTR = KEYSTR !! '/BEAM'; puthead
$ need Jy/beam for IMSTAT to work 
  KEYWORD 'BUNIT'; KEYSTR = 'Jy/BEAM'; puthead
  RETURN
FINISH

$setCONV() computes the appropriate convolution
$function for a GBT observation.   This function should
$be used after setting the CELLSIZE for SDIMG.
$The proceedure uses 5 arc second cellsize if cells is
$not set.
PROC SETCONV
  XTYPE = -16; YTYPE = -16
  GBTBEAM
  KEYWORD = 'BMAJ'; gethead
  BMAJ = KEYVAL(1)*3600.
  IF (CELLS(1) < 0.1) then cells = 5; END
  Y = BMAJ/CELLS(1) 
  XPARM = BMAJ, 1.55*BMAJ, 2.55*BMAJ, 2, 0
  YPARM = XPARM
  RETURN
FINISH
    
$Fix a common spelling error 
PROC TPYE(I)
  TYPE I
  RETURN
FINISH

$Fix a common spelling error 
PROC TYEP(I)
  TYPE I
  RETURN
FINISH

$Fix another common spelling error 
PROC GENT(I)
  GETN I
  RETURN
FINISH

$display a sequence of images on the TV
$Arguments are start and stop frequency channels
PROC TVLOOP(ISTART,IEND)
  FOR I=ISTART to IEND 
    TBLC(3) = I; TVLOD; TVLAB
    PRINT I
$Just enter <CR> to continue
    IF (PAUSEL > 0) THEN READ X; END
  END
  RETURN
FINISH

$ set PAUSEL = 1 to wait between channels
$ set PAUSEL = 0 for no pause in display loop
PAUSEL = 0
\end{verbatim}  
\end{document}
